\documentclass{article}

\usepackage[top=2.5cm, bottom=2.5cm, left=2cm, right=2cm]{geometry}
\usepackage{fancyhdr}
\lhead{\textsc{Numerical Algorithmic Project}}
\rhead{\textsc{Interpolation and Integration Methods}}
\renewcommand{\headrulewidth}{1px}
\lfoot{ \textsc{ENSEIRB-MATMECA}}
\rfoot{ \textsc{Computer Science - I1}}
\renewcommand{\footrulewidth}{1px}
\pagestyle{fancy}

\usepackage{wrapfig}
\usepackage{multicol}
\usepackage{textcomp}
\usepackage[T1]{fontenc}
\usepackage{graphicx}
\usepackage[english]{babel}
\usepackage{amsmath}
\usepackage{amssymb}
\usepackage{mathrsfs}
\usepackage[latin1]{inputenc}
\usepackage{xcolor}
\usepackage{algorithm}
\usepackage{algpseudocode}

\usepackage{setspace}
\doublespacing



%%%%%%%%%%%%%%%% Variables %%%%%%%%%%%%%%%%
\def\projet{5}
\def\titre{Interpolation and Integration Methods / Cubic Splines and Surface Interpolation}
\def\groupe{2}
\def\team{1}
\def\responsible{fmonjalet}
\def\secretary{tsanchez}
\def\others{rrasoldier, ylevif}

\begin{document}

%%%%%%%%%%%%%%%% Header %%%%%%%%%%%%%%%%
\noindent\begin{minipage}{\textwidth}
\vskip 0mm
\noindent
    { \begin{tabular}{p{7.5cm}}
        {\bfseries \sffamily
          Project n�\projet}
        \begin{center}{\itshape \titre}\end{center}
    \end{tabular}}
    \hfill 
    \fbox{\begin{tabular}{l}
        {~\hfill \bfseries \sffamily Group n�\groupe \hspace{3mm} Team n�\team \hfill~} \\[2mm] 
        Responsible : \responsible \\
        Secretary   : \secretary \\
        Programmers : \others \\
        Emails      : fmonjalet@enseirb-matmeca.fr,\\
        tsanchez@enseirb-matmeca.fr,\\
        rrasoldier@enseirb-matmeca.fr,\\
        ylevif@enseirb-matmeca.fr
    \end{tabular}}
    \vskip 4mm ~

    \parbox{\textwidth}{\small \textit{Abstract :} the goal of this project was to implement a basic model to represent the airflow around an airfoil. With that airflow model, it will be possible to obtain the pressure map of the air above and under the wing, in order to approximate the wing's lift (it's ability to keep the plane in the air). This is to be done in two steps : refining the airfoil into a sufficiently smooth curve, and then computing the pressure map using an integration method.
%      \begin{enumerate}
%      \item refining the airfoil into a sufficiently smooth curve;
%      \item computing the pressure map using an integration method.
%      \end{enumerate}
    }
    \vskip 1mm ~  
\end{minipage}

\section{Airfoil Refinement Using Cubic Splines}
The only airfoil representation available so far is a .dat file describing a cloud of points. So this file has to be converted into python exploitable data.

In order to be able to work on it using the python programming language, the .dat file information\footnote{The point cloud positions} has to be converted into a smooth curve. To obtain that curve, the points describing the upper part (named extrados) and the lower part (named intrados) of the airfoil will be interpolated using cubic splines.

\subsection{How to Interpolate a Cloud of Points with Cubic Splines}
To do this interpolation the best way possible, the method and formulas given by the \textit{Numerical Recipes in C} were used. We assume that a cloud of points $(x_i, y_i), \: i \in [0,N-1]$ is provided. There are two main parts of pre-calculus to be able to get the interpolation on any $x \in [x_0, x_{N-1}]$ point, given by the equation \ref{interp_form_1}:

\begin{equation}
\label{interp_form_1}
  interp(x)=A(x)*y_i + B(x)*y_{i+1} + C(x)*y''_i + D(x)*y''_{i+1}
\end{equation}

The first thing to carry out is to find the second derivatives of the curve at the points corresponding to $x_i, \: i \in [0,N-1]$. It can be done by solving a equation system. Indeed, all the $y_i, \: i \in [1,N-2]$ are linked by the following equation :

\begin{equation}
 \frac{x_i - x_{i-1}}{6}*y''_{i-1} + \frac{x_{i+1} - x_{i-1}}{3}*y''_{i} + \frac{x_{i+1} - x_i}{6}*y''_{i+1} = \frac{y_{i+1} - y_i}{x_{i+1} - x_i} - \frac{y_{i} - y_{i-1}}{x_{i} - x_{i-1}}
\end{equation}

There are multiple options for the values given to $y_0$ and $y_{N-1}$. We chose to set them to $0$. Once the values are set up, the system is solved. The whole process is summed up by algorithm \ref{interp_precalc}.

\begin{algorithm}
  \caption{Pre-calculus of the second derivatives for the cubic spline interpolation}
  \label{interp_precalc}
  \begin{algorithmic}[1]
    \Function{$cubic\_spline\_interpolation$}{$X:abscissa of the points, Y:ordinates of the points$}
    \State $N \gets size(X); \quad Y'' \gets null\_vector(N)$
    \State $Mat\_coeff \gets null\_matrix(N,N)$
    \State $Mat\_coeff[0, 0] \gets 1 ; \quad Mat\_coeff[N-1, N-1] \gets 1$
    \For{$i = 0 \to N-1$}
    \State{$Mat\_coeff[i, i-1] \gets (X[i] - X[i-1])/6$}
    \State{$Mat\_coeff[i, i] \gets (X[i+1] - X[i-1])/3$}
    \State{$Mat\_coeff[i, i+1] \gets (X[i+1] - X[i])/6$}
    \State{$Images[i] \gets (Y[i+1] - Y[i])/(X[i+1] - X[i]) - (Y[i] - Y[i-1])/(X[i] - X[i-1])$}
    \EndFor
    \State{$Y'' \gets solve\_linear\_system(Mat\_coeff, Images)$}
    \State \Return{$Y''$}
    \EndFunction
  \end{algorithmic}
\end{algorithm}


Once this system is solved, the values of $A(x)$, $B(x)$, $C(x)$ and $S(x)$ are computed, with the formulas given by the \textit{Numerical Recipes in C}, and $interp(x)$\footnote{The value of the cubic splines interpolation of the cloud of points in $x$.} is calculated.

The result is right, but one problem remains : the $interp(x)$ evaluation complexity is $\mathcal{O}(N^3)$ ($N$ being the number of points), because the system of equation has to be solved at each evaluation.

\subsection{Optimisation}
Now that everything is computed to calculate $interp(x)$, we notice that without optimisation, the equation system to find all the second derivatives will have to be solved every time $interp(x)$ is evaluated, and a lot of time will be lost. To avoid this problem, we chose to use functional programming, as shown in algorithm \ref{interp_calc}.

\begin{algorithm}
  \caption{Computation of the cubic spline interpolation function}
  \label{interp_calc}
  \begin{algorithmic}[1]
    \Function{$cubic\_splines\_interpolation\_precalc$}{$X:abscissa of the points, Y:ordinates of the points$}
    \State $N \gets size(X)$
    \State{$Y'' \gets cubic\_splines\_interpolation\_precalc(X,Y)$}
    \State \Return{$lambda(x) : cubic\_splines\_interpolation\_calc(X,Y,Y'',x)$}
    \EndFunction
  \end{algorithmic}
\end{algorithm}

The cubic\_splines\_interpolation\_precalc function pre computes the values of the second derivatives, and the cubic\_splines\_interpolation\_calc function that is returned only calculates the $A$, $B$, $C$ and $D$ factors fitting to the $x$ given in parameter, in order to return the interpolation value in $x$. The function returned memorizes the values of the second derivatives.

The final complexity of this process is: $\mathcal{O}(N^3)$ to obtain the interpolating function, and  $\mathcal{O}(N)$ to evaluate it. It is linear and not constant, because we have to find $i$ so that $x_i < x \leq  x_{i+1}$.

\subsection{Computing the Derivative}
For the second part of this project, a cubic spline derivative function had to be computed. The process won't be detailed here because it is mainly the same as the computation of the cubic spline function. The only thing that changes is when $A$, $B$, $C$ and $D$ are computed, we use the derivative of these coefficient with respect to $x$.

To sum up, all that have to be done is to compute the formal derivative of $A$, and all the other coefficient will be easily calculated by using it. This algorithm complexity is exactly the same as the previous one:  $\mathcal{O}(N^3)$ to obtain the interpolating function, and  $\mathcal{O}(N)$ to evaluate it

\subsection{Results}
Figure \ref{res_interp_1} shows the interpolation of the airfoil we used with the original points provided. Figure \ref{res_interp_2} shows the airfoil interpolation and its derivative.

\begin{figure}[h]
  \begin{minipage}{0.45\textwidth}
    \includegraphics[scale=0.4]{../src/interpolation_of_the_airfoil.png} 
    \caption{The airfoil interpolation with the original points provided}
    \label{res_interp_1}
  \end{minipage}
  %%%%%%%%%%%
  \hspace{0.1\textwidth}
  %%%%%%%%%%%
  \begin{minipage}{0.45\textwidth}
    \includegraphics[scale=0.4]{../src/interpolation_derivative.png}
    \caption{The airfoil interpolation and its derivative}
    \label{res_interp_2}
  \end{minipage}
\end{figure}


\section{Computing the Length of Plane Curves with Various Integration Methods}
Once a curve representing the airfoil is obtained, finding a way to compute the airfoil's length is the next step in the pressure map creation.
To do that, the equation \ref{int_gen_eq} will have to be computed with our values.

\begin{equation}
  \label{int_gen_eq}
  L([0;T]) = \int_{0}^{T} \sqrt{1 + f'(x)^2}dx
\end{equation}

Several integration methods will be used and compared.

Every method listed bellow follows the same steps to compute the integral value of a function on a interval and with a given number of intervals :

\begin{itemize}
\item the approximated integral value on the first interval is computed.
\item the integral on the next interval is computed and added this value to the previous
\item when the integral on the last interval has been calculated, the process is stopped
\end{itemize}

The only thing that will change is the method used to compute the integral value, therefore the convergence speed.

\subsection{Left Rectangle Integration Method}
The principle is the following:

\begin{itemize}
\item the value of $f(x)$ is computed with x the left bound of the interval
\item the surface of the rectangle having $f(x)$  as height and the interval's size as width is computed
\end{itemize}

Finally, the used formula is described by the equation \ref{int_left_rect}, for a function $f$, into the interval $[a,b]$ subdivided in $n$ intervals.

\begin{equation}
\label{int_left_rect}
I(f,a,b,n) = h * \sum_{k = 0}^{N-2} f(a+kh) \quad \text{with} \quad h = \frac{b-a}{n}
\end{equation}

\subsection{Right Rectangle Integration Method}
Same thing than the left rectangle method, except that the $f(x)$ value is the right bound of the interval. The formula is the equation \ref{int_right_rect}.

\begin{equation}
\label{int_right_rect}
I(f,a,b,n) = h * \sum_{k = 1}^{N-1} f(a+kh) \quad \text{with} \quad h = \frac{b-a}{n}
\end{equation}


\subsection{Middle Rectangle Integration Method}
Same thing than Left or Right rectangle method except that the $f(x)$ used is the middle of the interval. See equation \ref{int_mid}.

\begin{equation}
\label{int_mid}
I(f,a,b,n) = h * \sum_{k = 0}^{N-2} f(a+kh+\frac{h}{2}) \quad \text{with} \quad h = \frac{b-a}{n}
\end{equation}


\subsection{Trapezium Integration Method}
This integration method is a better alternatives to the previous method, even though the only difference is that both images of the left and right bounds are computed to construct a trapezium and calculate its surface. A good thing to notice is that we can deduce this method from the previous ones using the equation \ref{int_trap}.

\begin{equation}
\label{int_trap}
I(f,a,b,n) = I_{left\_rect} - \frac{h}{2}(f(a) - f(b)) \quad \text{with} \quad h = \frac{b-a}{n}
\end{equation}

Processing this way, $f$ is only evaluated once, instead of twice.


\subsection{Simpson Integration Method}
The problem with the previous methods is that they just give an average value of the function and do not take the possible variations inside the interval into account.

The Simpson method approximates the function by a polynomial on each interval. This way, the approximation follows the variations of the curve.

Thankfully, the real polynomial has not to be calculated, and the formula given by the equation \ref{int_simpson} calculates directly the integral of the second degree polynomial interpolating the curve.

\begin{equation}
\label{int_simpson}
I(f,a,b,n) = h * \sum_{k = 1}^{N-1} \frac{1/6} f(a+(k-1)h) + \frac{2/3} f(a+(k-0.5)h) + \frac{1/6} f(a+kh)  \quad \text{with} \quad h = \frac{b-a}{n}
\end{equation}


\subsection{Convergence Speed Comparison}
\begin{figure}[h]
  \begin{minipage}{0.45\textwidth}
    \includegraphics[scale=0.4]{../src/errors_integration.png}
    \caption{Error of the integration methods with respect to the number of points taken}
    \label{error_integration}
  \end{minipage}
  %%%%%%%%%%%
  \hspace{0.1\textwidth}
  %%%%%%%%%%%
  \begin{minipage}{0.45\textwidth}
    \includegraphics[scale=0.4]{../src/log_errors_integration.png}
    \caption{Error of the integration methods with respect to the number of points taken (logarithmic scales)}
    \label{log_error_integration}
  \end{minipage}
\end{figure}

As we can see on figure \ref{error_integration} and even more on figure \ref{log_error_integration} is that the Simpson method converges faster than the other ones.

With the logarithmic errors, we can determine more precisely, the convergence speed of the various methods. Indeed, the slope of the straight line provides the convergence speed, because powers become multiplicative factors in logarithmic scales. What can be seen is that the left and right rectangle method converge with a $\mathcal{O}(n)$ speed ($n$ being the number of iterations). The middle rectangle and trapezium method converge with a $\mathcal{O}(n^2)$ speed, and the Simpson method with a $\mathcal{O}(n^3)$ speed.

On these graphs, the function chosen is representative of the global results we found, but for the function $f(x) = \sqrt{1-x^2}$ on the interval $[-1,1]$ all the method converge with the same speed. We think that it's linked with the fact that this function is not derivable on $-1$ and $1$, but we had not the time to develop it.

In every case we saw, the Simpson method clearly seems to be the best.

The complexity of each method is $\mathcal{O}(n)$ in terms of $f$ calls ($n$ being the number of intervals).

\section{Computing the Pressure Map}
The airfoil length is required to compute the pressure of the air around the airfoil.
In order to compute this pressure, we have to model the airflow.

\subsection{Modeling the Laminar Airflow}
Even if the real airflow around an airfoil is not laminar, this model have been chosen to make the computation easier.
The laminar airflow around the airfoil can be modeled by using the formula below, each curve obtained represent the path followed by an air particle (see figure \ref{airflow}).

\begin{figure}

\end{figure}

Given the airfoil points interpolation function, each curve was computed for the extrados or the intrados with this formula : \[y = (1-\lambda)f(x)+\lambda\times3h\:\forall\lambda\in[0;1]\] where f is the result of the interpolation. Each curve corresponds to a value of $\lambda$.



\subsection{Computing the pressure variation map}
The pressure around the airfoil depends on the length of the path followed by the air particles. Indeed we assume that every air particle takes the same time to go from the beginning to the end of the airfoil, the more curved is the path the faster the particle will have to travel, and the greater dynamic pressure will be.

To build this graph, a lot of airflow paths are traced to reduce the gap between each curve, and each one is colorized with a colour corresponding to the pressure value on it with an arbitrary formula. The density of the curves and the gradient of color chosen provide a good representation of the pressure map (see figure \ref{pressure_map}).

\begin{figure}[h]
  \begin{minipage}{0.45\textwidth}
    \includegraphics[scale=0.4]{../src/airflow.png}
    \label{airflow}
    \caption{The airflow shape around the airfoil}
  \end{minipage}
  %%%%%%%%%%%
  \hspace{0.1\textwidth}
  %%%%%%%%%%%
  \begin{minipage}{0.45\textwidth}
    \includegraphics[scale=0.4]{../src/pressure_curves_simpson.png}
    \caption{Pressure map around the airfoil (the Simpson method was used)}
    \label{pressure_map}
  \end{minipage}
\end{figure}

\section{Conclusion}
This project has been an opportunity to understand the implication of every module involved in a modelling program. Indeed the integral computation used here could seem to be a small part of this project, but seeing the convergence speed comparison results, we realized that it is one of the more important part in term of performance and precision. We also saw the necessity to have a smooth derivable interpolation of the airfoil to obtain good results, which was possible thanks to cubic splines.
Moreover it still remains interesting to apply mathematical concepts to a very concrete purpose.


\end{document}
